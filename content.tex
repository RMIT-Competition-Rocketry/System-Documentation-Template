\pagestyle{ruled}

\phantomsection
\section*{Important Terms}
\addcontentsline{toc}{section}{Important Terms}
\begin{description}[leftmargin=8em,style=multiline]
    \item[Australis] Student developed flight computer hardware platform for high power rockets.
    \item[Australis Core] An internal component providing the base API and critical logic; stylised \verb|core|.
    \item[Australis Extra] An internal component providing modular systems that may be optionally included to extend core functionality; stylised \verb|extra|.
    \item[Australis Firmware] Flight computer firmware system for high power rockets; designed for, but not limited to, deployment on Australis targets.
    \item[Component] A collection of semantically related code groups and files within the Australis Firmware ecosystem.
    \item[Device] A hardware element external to the controller that provides additional functionality via a connected interface.
    \item[Driver] Software implementation of a device or peripheral interface.
    \item[Peripheral] A hardware element internal to the controller that provides important extensions to the feature set of its core processor.
    \item[Submodule] An isolated system packaged within \verb|extra| that extends system functionality to target source code. Submodules may only depend on the \verb|core| API.
    \item[Target] A hardware platform on which the Australis Firmware operates.
\end{description}

\phantomsection
\section*{Abbreviations}
\addcontentsline{toc}{section}{Abbreviations}
\begin{description}[leftmargin=3em,style=multiline]
    \item[A3\footnotemark] Aurora 3
    \item[API] Application Programming Interface
    \item[AV2] Australis Version 2
\end{description}
\footnotetext{{Also refers to version 1 of the Australis flight computer hardware.}}

\clearpage

\section{Section}
\begin{tcolorbox}[colback=gray!50,enhanced,boxrule=0pt,frame hidden]
    \paragraph{Note} This is a dark box
\end{tcolorbox}

\subsection{Subsection}
\begin{tcolorbox}[colback=gray!5, breakable, enhanced]
    \paragraph{Note}
    Footnotes do not work correctly in most environments, like here. To get around this we use \verb|\footnotemark| to define the location of the footnote marker, and \verb|\footnotetext{text}| later outside of the environment.\footnotemark
\end{tcolorbox}
\footnotetext{This is a footnote.}

Footnotes can be created normally with \verb|\footnote{text}| when called outside of an environment.\footnote{This is another footnote.}

Citations are created in IEEE style with Biblatex.\cite{example2}

\subsubsection{Subsubsection}

{% Wrap this table with parentheses like so to only apply captionsetup to the one table
\captionsetup{justification=raggedright,singlelinecheck=false}
\begin{table}[h]
  \begin{tabularx}{0.3\textwidth}{lX}
    Column 1 & Column 2\\
    \midrule
    Text     & Text\\
    \midrule
  \end{tabularx}
  \caption{Left aligned table with no row colouring}
\end{table}
}

\begin{table}[h]
  \centering
  \rowcolors{2}{}{green!25}
  \begin{tabularx}{0.3\textwidth}{lX}
    \toprule
    \rowcolor{blue!25}
    Column 1 & Column 2\\
    \midrule
    Text     & Text\\
    \midrule
    Text     & Text\\
    \midrule
    Text     & Text\\
    \bottomrule
  \end{tabularx}
  \caption{Center aligned table with row colouring}
\end{table}

\begin{table}[h]
  \rowcolors{2}{}{green!25}
  \centerline{\begin{tabularx}{\headwidth}{lX}
    \rowcolor{blue!25}
    Column 1 & Column 2\\
    \midrule
    Text     & Text\\
    \midrule
  \end{tabularx}}
  \caption{Full headwidth table extending past margins}
\end{table}

\noindent\lipsum[1][1-18]

\clearpage

\section{Memory and Bus Architecture}
\subsection{System Architecture}

In STM32F405xx/07xx and STM32F415xx/17xx, the main system consists of 32-bit 
multilayer AHB bus matrix that interconnects:
\begin{itemize}
    \item Eight masters:
    \begin{itemize}
        \item Cortex\textsuperscript{\textregistered}-M4 with FPU core I-bus, D-bus, and S-bus
        \item DMA1 memory bus
        \item DMA2 memory bus
        \item DMA2 peripheral bus
        \item Ethernet DMA bus
        \item USB OTG HS DMA bus
    \end{itemize}
    \item Seven slaves:
    \begin{itemize}
        \item Internal flash memory ICode bus
        \item Internal flash memory DCode bus
        \item Main internal SRAM1 (112 KB)
        \item Auxiliary internal SRAM2 (16 KB)
        \item AHB1 peripherals including AHB to APB bridges and APB peripherals
        \item AHB2 peripherals
        \item FSMC
    \end{itemize}
\end{itemize}

\noindent The bus matrix provides access from a master to a slave, enabling concurrent access and 
efficient operation even when several high-speed peripherals work simultaneously. 
The 64-Kbyte CCM (core coupled memory) data RAM is not part of the bus matrix and can be 
accessed only through the CPU. This architecture is shown in Figure 1.

\vfill{}
\begin{center}
EXAMPLE DOCUMENT \\ \small (totally not stolen from RM0090)
\end{center}
\vfill{}
